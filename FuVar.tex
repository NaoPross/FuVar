% !TeX program = xelatex
% !TeX encoding = utf8
% !TeX root = FuVar.tex

%% TODO: publish to CTAN
\documentclass[twocolumn, margin=normal]{tex/hsrzf}

%%%%%%%%%%%%%%%%%%%%%%%%%%%%%%%%%%%%%%%%%%%%%%%%%%%
% Packages

%% TODO: publish to CTAN
\usepackage{tex/hsrstud}
\usepackage{mathtools}

%% Language configuration
\usepackage{polyglossia}
\setdefaultlanguage{english}

%% License configuration
\usepackage[
    type={CC},
    modifier={by-nc-sa},
    version={4.0},
    lang={english},
]{doclicense}

%% Math
\usepackage{amsmath}
\usepackage{amsthm}

% Layout
\usepackage{enumitem}

%%%%%%%%%%%%%%%%%%%%%%%%%%%%%%%%%%%%%%%%%%%%%%%%%%%
% Metadata

\course{Elektrotechnik}
\module{FuVar}
\semester{Spring Semseter 2021}

\authoremail{naoki.pross@ost.ch}
\author{\textsl{Naoki Pross} -- \texttt{\theauthoremail}}

\title{\texttt{\themodule} Notes}
\date{\thesemester}

%%%%%%%%%%%%%%%%%%%%%%%%%%%%%%%%%%%%%%%%%%%%%%%%%%%
% Macros and settings

%% Theorems
\newtheoremstyle{fuvarzf} % name of the style to be used
  {\topsep}
  {\topsep}
  {}
  {0pt}
  {\bfseries}
  {.}
  { }
  { }

\theoremstyle{fuvarzf}
\newtheorem{theorem}{Theorem}
\newtheorem{proposition}{Proposition}
\newtheorem{method}{Method}
\newtheorem{definition}{Definition}
\newtheorem{lemma}{Lemma}
\newtheorem{remark}{Remark}

\DeclareMathOperator{\tr}{\mathrm{tr}}


\setlist[description]{
  format = { \normalfont\itshape }
}

%%%%%%%%%%%%%%%%%%%%%%%%%%%%%%%%%%%%%%%%%%%%%%%%%%%
% Document

\begin{document}

\maketitle
% \tableofcontents

\section{Preface}

These are just my personal notes of the \themodule{} course, and definitively
not a rigorously constructed mathematical text. The good looking \LaTeX{}
typesetting may trick you into thinking it is rigorous, but really, it is not.

\section{Derivatives of vector valued scalar functions}

\begin{definition}[Partial derivative]
  A vector values function \(f: \mathbb{R}^m\to\mathbb{R}\), with
  \(\vec{v}\in\mathbb{R}^m\), has a partial derivative with respect to \(v_i\)
  defined as
  \[
    \partial_{v_i} f(\vec{v})
      = f_{v_i}(\vec{v})
      = \lim_{h\to 0} \frac{f(\vec{v} + h\vec{e}_j) - f(\vec{v})}{h}
  \]
\end{definition}

\begin{proposition}
  Under some generally satisfied conditions (continuity of \(n\)-th order
  partial derivatives) Schwarz's theorem states that it is possible to swap
  the order of differentiation.
  \[
    \partial_x \partial_y f(x,y) = \partial_y \partial_x f(x,y)
  \]
\end{proposition}

\begin{definition}[Linearization]
  A function \(f: \mathbb{R}^m\to\mathbb{R}\) has a linearization \(g\) at
  \(\vec{x}_0\) given by
  \[
    g(\vec{x}) = f(\vec{x}_0) 
      + \sum_{i=1}^m \partial_{x_i} f(\vec{x}_0)(x_i - x_{i,0}) ,
  \]
  if all partial derviatives are defined at \(\vec{x}_0\).
\end{definition}

\begin{theorem}[Propagation of uncertanty]
  Given a measurement of \(m\) values in a vector \(\vec{x}\in\mathbb{R}^m\)
  with values given in the form \(x_i = \bar{x}_i \pm \sigma_{x_i}\), a linear
  approximation the error of a dependent variable \(y\) is computed with
  \[
    y = \bar{y} \pm \sigma_y \approx f(\bar{\vec{x}})
      \pm \sqrt{\sum_{i=1}^m \left(
        \partial_{x_i} f(\bar{\vec{x}}) \sigma_{x_i}\right)^2}
  \]
\end{theorem}

\begin{definition}[Gradient vector]
  The \emph{gradient} of a function \(f(\vec{x}), \vec{x}\in\mathbb{R}^m\) is a
  vector containing the derivatives in each direction.
  \[
    \grad f (\vec{x}) = \sum_{i=1}^m \partial_{x_i} f(\vec{x}) \vec{e}_i
      = \begin{pmatrix}
        \partial_{x_1} f(\vec{x}) \\
        \vdots \\
        \partial_{x_m} f(\vec{x}) \\
      \end{pmatrix}
  \]
\end{definition}

\begin{definition}[Directional derivative]
  A function \(f(\vec{x})\) has a directional derivative in direction
  \(\vec{r}\) (with \(|\vec{r}| = 1\)) given by
  \[
    \frac{\partial f}{\partial\vec{r}} = \nabla_\vec{r} f = \vec{r} \dotp \grad f
  \]
\end{definition}

\begin{theorem}
  The gradient vector always points towards \emph{the direction of steepest ascent}.
\end{theorem}

\subsection{Methods for maximization and minimization problems}

\begin{method}[Find stationary points]
  Given a function \(f: D \subseteq \mathbb{R}^m \to \mathbb{R}\), to
  find its maxima and minima we shall consider the points
  \begin{itemize}
    \item that are on the boundary of the domain \(\partial D\),
    \item where the gradient \(\grad f\) is not defined,
    \item that are stationary, i.e. where \(\grad f = \vec{0}\).
  \end{itemize}
\end{method}

\begin{method}[Determine the type of stationary point for 2 dimensions]
  Given a scalar function of two variables \(f(x,y)\) and a stationary point
  \(\vec{x}_s\) (where \(\grad f(\vec{x}_s) = \vec{0}\)), we define the
  \emph{discriminant}
  \[
    \Delta = \partial_x^2 f \partial_y^2 f - \partial_y \partial_x f
  \]
  \begin{itemize}
    \item if \(\Delta > 0\) then \(\vec{x}_s\) is an extrema, if \(\partial_x^2
      f(\vec{x}_s) < 0\) it is a maximum, whereas if \(\partial_x^2
      f(\vec{x}_s) > 0\) it is a minimum;

    \item if \(\Delta < 0\) then \(\vec{x}_s\) is a saddle point;

    \item if \(\Delta = 0\) we need to analyze further.
  \end{itemize}
\end{method}

\begin{remark}
  The previous method is obtained by studying the second directional derivative
  \(\nabla_\vec{r}\nabla_\vec{r} f\) at the stationary point in direction of a
  vector \(\vec{r} = \vec{e}_1\cos(\alpha) + \vec{e}_2\sin(\alpha)\)
\end{remark}

\begin{definition}[Hessian matrix]
  Given a function \(f: \mathbb{R}^m \to \mathbb{R}\), the square matrix whose
  entry at the \(i\)-th row and \(j\)-th column is the second derivative of
  \(f\) first with respect to \(x_j\) and then to \(x_i\) is know as the
  \emph{Hessian} matrix.
  \(
    \left(\mtx{H}_f\right)_{i,j} = \partial_{x_i}\partial_{x_j} f
  \)
  or
  \[
    \mtx{H}_f = \begin{pmatrix}
      \partial_{x_1}\partial_{x_1} f & \cdots & \partial_{x_1}\partial_{x_m} f \\
      \vdots & \ddots & \vdots \\
      \partial_{x_m}\partial_{x_1} f & \cdots & \partial_{x_m}\partial_{x_m} f \\
    \end{pmatrix}
  \]
  Because (almost always) the order of differentiation
  does not matter, it is a symmetric matrix.
\end{definition}

\begin{method}[Determine the type of stationary point in higher dimensions]
  Given a scalar function of two variables \(f(x,y)\) and a stationary point
  \(\vec{x}_s\) (where \(\grad f(\vec{x}_s) = \vec{0}\)), we compute the
  Hessian matrix \(\mtx{H}_f(\vec{x}_s)\). Then we compute its eigenvalues
  \(\lambda_1, \ldots, \lambda_m\) and
  \begin{itemize}
    \item if all \(\lambda_i > 0\), the point is a minimum;
    \item if all \(\lambda_i < 0\), the point is a maximum;
    \item if there are both positive and negative eigenvalues,
      it is a saddle point.
  \end{itemize}
  In the other cases, when there are \(\lambda_i \leq 0\) and/or \(\lambda_i
  \geq 0\) further analysis is required.
\end{method}

\begin{remark}
  Recall that to compute the eigenvalues of a matrix, one must solve the
  equation \((\mtx{H} - \lambda\mtx{I})\vec{x} = \vec{0}\). Which can be done
  by solving the characteristic polynomial \(\det\left(\mtx{H} -
  \lambda\mtx{I}\right) = 0\) to obtain \(\dim(\mtx{H})\) \(\lambda_i\), which
  when plugged back in result in a overdetermined system of equations.
\end{remark}

\begin{method}[Quickly find the eigenvalues of a \(2\times 2\) matrix]
  Let
  \[
    m = \frac{1}{2}\tr \mtx{H} = \frac{a + d}{2}
    \text{  and  }
    p = \det\mtx{H} = ad - bc ,
  \]
  then
  \[
    \lambda = m \pm \sqrt{m^2 - p} .
  \]
\end{method}

\begin{method}[Search for a constrained extremum in 2 dimensions]
  Let \(n(x,y) = 0\) be a constraint in the search of the extrema of a function
  \(f: D \subseteq \mathbb{R}^2 \to \mathbb{R}\). To find the extrema we look for
  points
  \begin{itemize}
    \item on the boundary \(\vec{u} \in \partial D\) where \(n(\vec{u}) = 0\);

    \item \(\vec{u}\) where the gradient either does not exist or is
      \(\vec{0}\), and satisfy \(n(\vec{u}) = 0\);

    \item that solve the system of equations
      \[
        \begin{cases}
          \partial_x f(\vec{u}) \cdot \partial_y n(\vec{u})
            = \partial_y f(\vec{u}) \cdot \partial_x n(\vec{u}) \\
          n(\vec{u}) = 0
        \end{cases}
      \]
  \end{itemize}
\end{method}

\begin{method}[%
    Search for a constrained extremum in higher dimensions,
    method of Lagrange multipliers]
  We wish to find the extrema of \(f: D \subseteq \mathbb{R}^m \to \mathbb{R}\)
  under \(k < m\) constraints \(n_1 = 0, \cdots, n_k = 0\). For that we consider
  the following points:
  \begin{itemize}
    \item Points on the boundary \(\vec{u} \in \partial D\) that satisfy
      \(n_i(\vec{u}) = 0\) for all \(1 \leq i \leq k\), 

    \item Points \(\vec{u} \in D\) where either
      \begin{itemize}
        \item any of \(\grad f, \grad n_1, \ldots, \grad n_k\) do not exist, or
        \item \(\grad n_1, \ldots, \grad n_k\) are linearly \emph{dependent},
      \end{itemize}
      and that satisfy \(0 = n_1(\vec{u}) = \ldots = n_k(\vec{u})\).

    \item Points that solve the system of \(m+k\) equations
      \[
        \begin{dcases}
          \grad f(\vec{u}) = \sum_{i = 1}^k \lambda_k \grad n_i(\vec{u})
            & (m\text{-dimensional}) \\
          n_i(\vec{u}) = 0  & \text{ for } 1 \leq i \leq k
        \end{dcases}
      \]
      The \(\lambda\) values are known as \emph{Lagrange multipliers}.
  \end{itemize}
\end{method}

\section*{License}
\doclicenseText

\begin{center}
  \doclicenseImage
\end{center}

\end{document}
